\documentclass[12pt, titlepage]{article}

\usepackage{booktabs}
\usepackage{tabularx}
\usepackage{hyperref}
\usepackage{lscape}
\usepackage{longtable}
\usepackage{graphicx}
\hypersetup{
    colorlinks,
    citecolor=black,
    filecolor=black,
    linkcolor=red,
    urlcolor=blue
}
\usepackage[round]{natbib}

\input{../Comments}
%% Common Parts

\newcommand{\progname}{Mechatronics} % PUT YOUR PROGRAM NAME HERE
\newcommand{\authname}{Team \#20, OpenASL
\\ Robert Zhu zhul49
\\ Zifan Meng mengz17
\\ Jiahui Chen chenj194
\\ Kelvin Huynh huynhk12
\\ Runze Zhu zhur25
\\ Mirza Nafi Hasan hasanm21} % AUTHOR NAMES                  

\usepackage{hyperref}
    \hypersetup{colorlinks=true, linkcolor=blue, citecolor=blue, filecolor=blue,
                urlcolor=blue, unicode=false}
    \urlstyle{same}
                                


\begin{document}

\title{Verification and Validation Report: \progname} 
\author{\authname}
\date{\today}
	
\maketitle

\pagenumbering{roman}

\section{Revision History}

\begin{tabularx}{\textwidth}{p{3cm}p{2cm}X}
\toprule {\bf Date} & {\bf Version} & {\bf Notes}\\
\midrule
Date 1 & 1.0 & Notes\\
Date 2 & 1.1 & Notes\\
\bottomrule
\end{tabularx}

~\newpage

\section{Symbols, Abbreviations and Acronyms}

\renewcommand{\arraystretch}{1.2}
\begin{tabular}{l l} 
  \toprule		
  \textbf{symbol} & \textbf{description}\\
  \midrule 
  T & Test\\
  \bottomrule
\end{tabular}\\

\wss{symbols, abbreviations or acronyms -- you can reference the SRS tables if needed}

\newpage

\tableofcontents

\listoftables %if appropriate

\listoffigures %if appropriate

\newpage

\pagenumbering{arabic}

This document ...

\section{Functional Requirements Evaluation}
Section 5 from V&V Plan \\
Hardware: \\
1. Camera Detection \\
2. Raspberry Pi \\
\\~\\
Machine Learning: \\
1. Joint Recognition \\
2. Display coordinates in space \\
3. Detecting multiple pairs of hands \\
4. Detecting multiple pairs of hands \\
5. Trainable Model  \\
\\~\\
Real-time Data Processing \\ 
1. Process data in real-time \\

\newpage
\centerline{HARDWARE FUNCTIONAL REQUIREMENTS}
\centerline{The reference for ASL alphabet is Table \#}

\begin{landscape}
\renewcommand{\arraystretch}{1.2}
\noindent \begin{longtable}{p{0.05\linewidth}|p{0.17\linewidth}|p{0.2\linewidth}|p{0.15\linewidth}|p{0.15\linewidth}|p{0.2\linewidth}|p{0.05\linewidth}}
\hline
\textbf{ID} & \textbf{Description} & \textbf{Requirement Reference} & \textbf{Input} & \textbf{Expected Output} & \textbf{Actual Output} & \textbf{Result}\\
\hline
T1 & Camera is set up on Raspberry Pi & CFR1 & Raspistill command to take a picture & A picture & A picture & Pass\\ \hline
T2 & Real-time video is captured and displayed on screen & CFR1, CFR2 & Views in front of the camera & Views in front of the camera are displayed & Views in front of the camera are displayed & Pass\\ \hline
T3 & Testing hand detection at a distance & CFR1 & hand gesture for “d” & d & d & Pass\\ \hline
T4 & Testing if different webcams or cameras impact a sentence & CFR1, CFR2 & Sign the sentence “how do you do” alphabetically through 5 different cameras & & & Pass\\ \hline
T5 & Testing hand detection with multiple hands & CFR3 & hand gestures for “d” & d & d & Pass\\ \hline
T6 & Testing hand detection for similar gestures & CFR1 & hand gesture for “m” & m & n & Fail\\ \hline
T7 & Testing hand detection for motion (no input) & CFR1 & Static hand gestures (no motions) & No output & z/d & Fail\\ \hline
T8 & Testing hand detection for motion & CFR1 & hand motion for “z” & z & z & Pass\\ \hline
T9 & Testing hand detection for hand at the edges of the camera detection area & CFR1 & hand gesture for “d” & d & d & Pass\\ \hline
T10 & Mode Selection & N/A & number “2” on keyboard & System goes into “Training Mode” & System goes into “Training Mode” & Pass\\ \hline
T11 & String display for one hand gesture & N/A & Any hand gesture except for the gesture for “Speak” & Text representing the hand gesture is displayed & Text representing the hand gesture is displayed & Pass\\ \hline
T12 & String display for a series of hand gestures (slow) & N/A & A series of hand gestures (except for “Speak”) with a pause of 2 seconds & Text representing each hand gesture is displayed as a string & Text representing each hand gesture is displayed as a string & Pass\\ \hline
T13 & String display for a series of hand gestures (fast) & N/A & A series of hand gestures (except for “Speak”) with a pause of  less than 2 seconds & Text representing each hand gesture is displayed as a string & Some gesture’s tests are not displayed & Fail\\ \hline
T14 & Modifying string display & N/A & Pressing “Backspace” or “Space” & “Backspace” deletes a character, “Space” adds a space & “Backspace” deletes a character, “Space” adds a space & Pass\\
T15 & Testing hand detection for a series of hand gestures (fast) & CFR1 & A series of hand gestures performed in a very fast speed & Letters for corresponding hand gestures & Some letters are missing & Fail (need to increase fps)
\hline
\caption{HARDWARE FUNCTIONAL REQUIREMENTS}
\end{longtable}
\end{landscape}

\newpage
\centerline{MACHINE LEARNING FUNCTION REQUIREMENTS}

\begin{landscape}
\renewcommand{\arraystretch}{1.2}
\noindent \begin{longtable}{p{0.05\linewidth}|p{0.17\linewidth}|p{0.1\linewidth}|p{0.15\linewidth}|p{0.25\linewidth}|p{0.25\linewidth}|p{0.05\linewidth}}
\hline
\textbf{ID} & \textbf{Description} & \textbf{Requirement Reference} & \textbf{Input} & \textbf{Expected Output} & \textbf{Actual Output} & \textbf{Result}\\
\hline
M1 & Testing for joint tracking when hiding joints & MLFR1, MLFR5, NFR2 & Hand Gesture for “m” and “n” (covering thumb) & Able to recognize hidden joints & Able to recognize hidden joints & Pass\\ \hline
M2 & Testing for joint tracking when overlapping hands & MLFR1, MLFR3, MLFR5 & Hand Gesture for “S”, “M”, “N”, “R” & Able to separate different hand joints from each other & Able to separate different hand joints from each other & Pass\\ \hline
M3 & Testing if joint lines are properly aligned with the user’s joints and move accordingly at the center & MLFR1, MLFR6, NFR2 & Moving hand from one side of the screen to the other in rapid succession & Able to overlay joint lines on user’s hand continually and is centered on the hand & Able to overlay joint lines on user’s hand continually and is centered on the hand & Pass\\ \hline
M4 & Testing if a joint overlay will be placed on more than two hands & MLFR1, MLFR3, NFR2 & Having a third hand in the frame after the initial two & Unable to detect the third hand & Unable to detect the third hand & Pass\\ \hline
M5 & Testing if detected joints are from one individual (the user) & MLFR1, NFR1, NFR3 & Have two people with one hand each in the frame & Detects the hand from one person as opposed to two & Detects both the hands of both people & Fail\\ \hline
M6 & Testing if the coordinates (x,y) of each joint is accurately recorded & MLFR2 & Repeatedly recording the gesture “a” at the center of the screen & 0,0,0,-0.374045802,-0.038167939,-0.709923664,-0.381679389,-0.824427481,-0.72519084,-0.824427481,-1,-0.541984733,-0.65648855,-0.671755725,-0.86259542,-0.564885496,-0.610687023,-0.465648855,-0.389312977,-0.328244275,-0.72519084,-0.480916031,-0.885496183,-0.366412214,-0.541984733,-0.251908397,-0.305343511,-0.122137405,-0.763358779,-0.251908397,-0.870229008,-0.160305344,-0.519083969,-0.076335878,-0.282442748,0.106870229,-0.770992366,-0.045801527,-0.847328244,-0.015267176,-0.603053435,0.038167939,-0.419847328 & 0,0,0,-0.374045802,-0.038167939,-0.709923664,-0.381679389,-0.824427481,-0.72519084,-0.824427481,-1,-0.541984733,-0.65648855,-0.671755725,-0.86259542,-0.564885496,-0.610687023,-0.465648855,-0.389312977,-0.328244275,-0.72519084,-0.480916031,-0.885496183,-0.366412214,-0.541984733,-0.251908397,-0.305343511,-0.122137405,-0.763358779,-0.251908397,-0.870229008,-0.160305344,-0.519083969,-0.076335878,-0.282442748,0.106870229,-0.770992366,-0.045801527,-0.847328244,-0.015267176,-0.603053435,0.038167939,-0.419847328 & Pass\\ \hline
M7 & Testing if the coordinates (x,y) of each joint is accurately recorded for two handed gestures & MLFR2, MLFR3 & Repeatedly recording the gesture “F” at the center of the screen & 6,0,0,0.121428571,-0.232142857,0.357142857,-0.328571429,0.578571429,-0.342857143,0.728571429,-0.357142857,0.328571429,-0.310714286,0.660714286,-0.314285714,0.85,-0.307142857,1,-0.292857143,0.346428571,-0.121428571,0.628571429,-0.171428571,0.560714286,-0.203571429,0.457142857,-0.196428571,0.342857143,0.042857143,0.571428571,-0.039285714,0.496428571,-0.071428571,0.403571429,-0.064285714,0.339285714,0.175,0.521428571,0.078571429,0.45,0.042857143,0.357142857,0.057142857 & 6,0,0,0.121428571,-0.232142857,0.357142857,-0.328571429,0.578571429,-0.342857143,0.728571429,-0.357142857,0.328571429,-0.310714286,0.660714286,-0.314285714,0.85,-0.307142857,1,-0.292857143,0.346428571,-0.121428571,0.628571429,-0.171428571,0.560714286,-0.203571429,0.457142857,-0.196428571,0.342857143,0.042857143,0.571428571,-0.039285714,0.496428571,-0.071428571,0.403571429,-0.064285714,0.339285714,0.175,0.521428571,0.078571429,0.45,0.042857143,0.357142857,0.057142857 & Pass\\ \hline
M8 & Testing if gestures that require movement are able to be recognized (motion gestures) & MLFR4, MLFR6 & Signing“j” and “z”  & j z & j z & Pass\\ \hline
M9 & Test if TTS string creation speed can keep up with signing speed at one gesture per second & MLFR6, MLFR4, NFR1 & Signing letters in sequence at one gesture per second & a b c d e f g & a b c d e f g & Pass\\ \hline
M10 & Test if a .tflite file can be generated from the CSV files & MLFR5, NFR5 & A CSV file with data points from different ASL gestures & A .tflite file that can be used to  recognize the gestures that were recorded & A .tflite file that can be used to  recognize the gestures that were recorded & Pass\\ \hline
M11 & Testing if retraining by adding new data points can change recognition & MLFR7, NFR1, NFR5 & Adding 50 accurate data points to the gesture “Hello” & The accuracy prediction increases & The accuracy prediction decrease from 60\% to 80\% & Pass\\ \hline
M12 & Test model accuracy by signing different sequences of gestures / introducing variance into the system & MLFR4, NFR1 & Sign letters in sequence of a,b,c,d then sign with d, f, z, j & a,b,c,d d,f,z,j with 100\% accuracy & a,b,c,d d,f,z,j & Pass\\ \hline
M13 & Testing for gesture variation based on user habits through retraining & MLFR7, NFR1, NFR3, NFR7 & Retraining the model with a different method of signing “Hello” & Hello & Hello & Pass\\
\hline
\caption{MACHINE LEARNING FUNCTION REQUIREMENTS}
\end{longtable}
\end{landscape}

\newpage
\centerline{Real-time Data Processing}

\begin{landscape}
\renewcommand{\arraystretch}{1.2}
\noindent \begin{longtable}{p{0.05\linewidth}|p{0.17\linewidth}|p{0.1\linewidth}|p{0.15\linewidth}|p{0.25\linewidth}|p{0.25\linewidth}|p{0.05\linewidth}}
\hline
\textbf{ID} & \textbf{Description} & \textbf{Requirement Reference} & \textbf{Input} & \textbf{Expected Output} & \textbf{Actual Output} & \textbf{Result}\\
\hline
1 & test process data in real-time & RDP1 & hand gestures for “d” and “a” & Output the corresponding letters “da” right after the hand gestures & Letters “da” were detected and displayed right after the hand gestures & Pass\\ \hline
2 & Testing if the relative coordinates (x,y) is written to the CSV file & RDP1, NFR5 & Hand gesture for “a” & Coordinates with identifier “0” (identifier for the letter “a”) are written to the CSV file & Coordinates with identifier “0” were written to the CSV file & Pass\\ \hline
3 & Testing if the point history coordinates (x,y) is written to the CSV file & RDP1, NFR5 & Hand gesture for “j” & Multiple coordinates with identifier “9” (identifier for the letter “j”) are written to the CSV file & Multiple coordinates with identifier “9” get written to the CSV file & Pass\\ \hline
4 & Text to Speech & RDP2 & Hand gesture for “Speak” & Current string is read out and then cleared & Current string is read out and then cleared & Pass\\ \hline
5 & Text-to-speech in real-time for individual letters & RDP2 & Hand gestures for “a”, “b” and “c” & Audio output for letters “a”, “b” and “c” & Audio output for letters “a”, “b” and “c” & Pass\\ \hline
6 & Text-to-speech in real-time for sentence & RDP2 & hand gesture for “I love you” & Audio output for “I love you” & Audio output for “I love you” & Pass\\ \hline
7 & Switching from translating mode to retraining mode stop detecting hand gestures & N/A & Pressing either 2 or 3 & The interface no longer tries to record hand motion & The interface no longer tries to record hand motion & Pass
\hline
\caption{Real-time Data Processing}
\end{longtable}
\end{landscape}

\section{Nonfunctional Requirements Evaluation}
\newpage
\centerline{accuracy, usability, portability, cultural}

\begin{landscape}
\renewcommand{\arraystretch}{1.2}
\noindent \begin{longtable}{p{0.05\linewidth}|p{0.17\linewidth}|p{0.1\linewidth}|p{0.15\linewidth}|p{0.25\linewidth}|p{0.25\linewidth}|p{0.05\linewidth}}
\hline
\textbf{ID} & \textbf{Description} & \textbf{Requirement Reference} & \textbf{Input} & \textbf{Expected Output} & \textbf{Actual Output} & \textbf{Result}\\
\hline
1 & Test if GUI is displayed on screen & N/A & Program is started and camera is turned on & The resolution, FPS, mode, and current text are displayed on screen & The resolution, FPS, mode, and current text are displayed on screen & Pass\\ \hline
2 & Test if gestures are not written to string when in training mode & N/A & Program is started, in training mode, and gestures are being performed & Nothing is being added to the string & Nothing is added to the string & Pass\\ \hline
3 & Test if output is accurate for variations in user gestures & NFR7 & Trying three variations of “Hello” & Hello & Hello & Pass\\ \hline
4 & Usability: the ease of use of a user without the knowledge of ASL &  & Instructions and example hand gestures are provided to the user & The user should know how to use the ASL device and can input some sample ASL words after reading the instructions & The user is able to use the ASL device and input some sample ASL words after reading the instructions & Pass\\ \hline
5 & Test if the Raspberry Pi can capture and translate ASL in real time &  & Program is started on the Raspberry Pi & The Raspberry Pi should be able to use the camera to detect and translate ASL in real time & The Raspberry Pi camera does not display the video with an adequate frame rate, making translation undoable & Fail\\
\hline
\caption{accuracy, usability, portability, cultural}
\end{longtable}
\end{landscape}

\subsection{Usability}
		
\subsection{Performance}

\subsection{etc.}
	
\section{Comparison to Existing Implementation}	

This section will not be appropriate for every project.

\section{Unit Testing}

\section{Changes Due to Testing}

\wss{This section should highlight how feedback from the users and from 
the supervisor (when one exists) shaped the final product.  In particular 
the feedback from the Rev 0 demo to the supervisor (or to potential users) 
should be highlighted.}

\section{Automated Testing}
		
\section{Trace to Requirements}
		
\section{Trace to Modules}		

\section{Code Coverage Metrics}

\bibliographystyle{plainnat}
\bibliography{../../refs/References}

\newpage{}
\section*{Appendix --- Reflection}

The information in this section will be used to evaluate the team members on the
graduate attribute of Reflection.  Please answer the following question:

\begin{enumerate}
  \item In what ways was the Verification and Validation (VnV) Plan different
  from the activities that were actually conducted for VnV?  If there were
  differences, what changes required the modification in the plan?  Why did
  these changes occur?  Would you be able to anticipate these changes in future
  projects?  If there weren't any differences, how was your team able to clearly
  predict a feasible amount of effort and the right tasks needed to build the
  evidence that demonstrates the required quality?  (It is expected that most
  teams will have had to deviate from their original VnV Plan.)
\end{enumerate}

\end{document}