\documentclass[12pt, titlepage]{article}

\usepackage{booktabs}
\usepackage{tabularx}
\usepackage{hyperref}
\usepackage{lscape}
\usepackage{longtable}
\usepackage{graphicx}
\hypersetup{
    colorlinks,
    citecolor=black,
    filecolor=black,
    linkcolor=red,
    urlcolor=blue
}
\usepackage[round]{natbib}

\input{Comments}
%% Common Parts

\newcommand{\progname}{Mechatronics} % PUT YOUR PROGRAM NAME HERE
\newcommand{\authname}{Team \#20, OpenASL
\\ Robert Zhu zhul49
\\ Zifan Meng mengz17
\\ Jiahui Chen chenj194
\\ Kelvin Huynh huynhk12
\\ Runze Zhu zhur25
\\ Mirza Nafi Hasan hasanm21} % AUTHOR NAMES                  

\usepackage{hyperref}
    \hypersetup{colorlinks=true, linkcolor=blue, citecolor=blue, filecolor=blue,
                urlcolor=blue, unicode=false}
    \urlstyle{same}
                                


\begin{document}

\title{Verification and Validation Report: \progname} 
\author{\authname}
\date{\today}
	
\maketitle

\pagenumbering{roman}

\section{Revision History}

\begin{tabularx}{\textwidth}{p{3cm}p{2cm}X}
\toprule {\bf Date} & {\bf Version} & {\bf Notes}\\
\midrule
Date 1 & 1.0 & Notes\\
Date 2 & 1.1 & Notes\\
\bottomrule
\end{tabularx}

~\newpage

\section{Symbols, Abbreviations and Acronyms}

\renewcommand{\arraystretch}{1.2}
\begin{tabular}{l l} 
  \toprule		
  \textbf{symbol} & \textbf{description}\\
  \midrule 
  T & Test\\
  \bottomrule
\end{tabular}\\

\wss{symbols, abbreviations or acronyms -- you can reference the SRS tables if needed}

\newpage

\tableofcontents

\listoftables %if appropriate

\listoffigures %if appropriate

\newpage

\pagenumbering{arabic}

This document ...

\section{Functional Requirements Evaluation}

\section{Nonfunctional Requirements Evaluation}

\subsection{Usability}
		
\subsection{Performance}

\subsection{etc.}
	
\section{Comparison to Existing Implementation}	

This section will not be appropriate for every project.

\section{Unit Testing}

\section{Changes Due to Testing}

\wss{This section should highlight how feedback from the users and from 
the supervisor (when one exists) shaped the final product.  In particular 
the feedback from the Rev 0 demo to the supervisor (or to potential users) 
should be highlighted.}

\section{Automated Testing}
		
\section{Trace to Requirements}
		
\section{Trace to Modules}		

\section{Code Coverage Metrics}

\newpage
\centerline{Tests for Motion Tracking Module}

\renewcommand{\arraystretch}{1.2}
\noindent \begin{longtable}{p{0.05\linewidth}|p{0.17\linewidth}|p{0.11\linewidth}|p{0.15\linewidth}|p{0.15\linewidth}|p{0.15\linewidth}|p{0.08\linewidth}}
\hline
\textbf{ID} & \textbf{Description} & \textbf{Req Ref} & \textbf{Input} & \textbf{Expected Output} & \textbf{Actual Output} & \textbf{Result}\\
\hline
A1 & Testing for joint tracking when hiding joints & MLFR1, MLFR5, NFR2 & Hand Gesture for “m” and “n” (covering thumb) & Able to recognize hidden joints & Able to recognize hidden joints & Pass\\ \hline
A2 & Testing hand detection for hand at the edges of the camera detection area & CFR1 & Hand gesture for “a”, “b”, “c” & a b c & a b c & Pass\\ \hline
A3 & Testing if joint lines are properly aligned with the user’s joints and move accordingly at the center & MLFR1, MLFR6, NFR2 & Moving hand from one side of the screen to the other in rapid succession & Able to overlay joint lines on user’s hand continually and is centered on the hand & Able to overlay joint lines on user’s hand continually and is centered on the hand & Pass\\ \hline
A4 & Testing if a joint overlay will be placed on more than two hands & MLFR1, MLFR3, NFR2 & Having a third hand in the frame after the initial two & Unable to detect the third hand & Unable to detect the third hand & Pass\\ \hline
A5 & Testing if detected joints are from one individual (the user) & MLFR1, NFR1, NFR3 & Have two people with one hand each in the frame & Detects the hand from one person as opposed to two & Detects both the hands of both people & Fail\\ \hline
A6 & Testing hand detection at a distance of 2 m & CFR1 & Hand gesture for “a”, “b”, “c” & a b c & a b c & Pass\\ \hline
A7 & Testing hand detection with multiple hands & CFR1 & Hand gestures for “z”, “x”, “y” & z x y & z x y & Pass\\ \hline
A8 & Testing for joint tracking when overlapping hands & MLFR1,MLFR3, MLFR5 & Hand Gesture for “S”, “M”, “N”, “R” & Able to separate different hand joints from each other & Able to separate different hand joints from each other & Pass\\ \hline
A9 & Switching from translating mode to training mode stop detecting hand gestures & N/A & Pressing either 2 or 3 & The interface no longer tries to record hand motion & The interface no longer tries to record hand motion & Pass\\ \hline
A10 & Testing for precision tracking & MLFR1 & Making small rotations and tremors & The joint overlay makes small movements & The joint overlay makes small movements & Pass\\ \hline
A11 & Testing for gesture recognition if the hands hand in placing with different angles & CFR1 & Hand gestures for “a”, “b”, “c” with different angles for the position of the hand & “a”, “b”, “c” & “a”, “b”, “c” & Pass\\ \hline
A12 & Testing for occlusion handling & MLFR1 & Partially hiding half of the hand behind a desk & The joint overlay is able to predict the rest of the hand & Joints overlay becomes disjointed and stretches & Fail\\ \hline
A13 & Testing the durability for accuracy and reliability & NFR1, MLFR1 & Keeping the program open for over an hour and testing for similar results & The joint overlay works as intended & The frame rate decreased leading to poor performance & Fail\\ \hline
A14 & User testing for different hand sizes and shapes & MLF1 & Using different people’s hands to test the accuracy of the string “a”, “”b”, “c”, “d” & Able to translate a b c d everytime & Able to translate a b c d everytime & Pass
\hline
\caption{Tests for Motion Tracking Module}
\end{longtable}

\newpage
\centerline{Tests for Coordinate Normalization Module}

\renewcommand{\arraystretch}{1.2}
\noindent \begin{longtable}{p{0.05\linewidth}|p{0.17\linewidth}|p{0.11\linewidth}|p{0.15\linewidth}|p{0.15\linewidth}|p{0.15\linewidth}|p{0.08\linewidth}}
\hline
\textbf{ID} & \textbf{Description} & \textbf{Req Ref} & \textbf{Input} & \textbf{Expected Output} & \textbf{Actual Output} & \textbf{Result}\\
\hline
B1 & Testing if different webcams or cameras impact coordinates at the same position & CFR1, CFR2 & Sign the sentence “how do you do” alphabetically through 5 different cameras & The same set of coordinates for all 5 & The same set of coordinates for all 5 & Pass\\ \hline
B2 & Testing if the coordinates (x,y) of each joint is accurately recorded & MLFR2 & Repeatedly recording the gesture “a” at the center of the screen & The same set of coordinates should be written to CSV file every time the gesture is recorded & The same set of coordinates should be written to CSV file every time the gesture is recorded & Pass\\ \hline
B3 & Testing if the coordinates (x,y) of each joint is accurately recorded for two handed gestures & MLFR3, MLFR2 & Repeatedly recording the gesture “F” at the center of the screen & The same set of coordinates should be written to CSV file every time the gesture is recorded & The same set of coordinates should be written to CSV file every time the gesture is recorded & Pass\\ \hline
B4 & Testing for range normalization between [-1,1] & MLFR2 & Testing the joints at the edge of the frame & No coordinate recorded exceeds [-1, 1] & No coordinate recorded exceeds [-1, 1] & Pass\\ \hline
B5 & Testing for scaling normalization for hand size to be consistent & MLFR2 & Testing using different sizes to hands & All coordinates recorded from each set of hands are generally the same & All coordinates recorded from each set of hands are generally the same & Pass
\hline
\caption{Tests for Coordinate Normalization Module}
\end{longtable}

\newpage
\centerline{Tests for Coordinate Export Module}

\renewcommand{\arraystretch}{1.2}
\noindent \begin{longtable}{p{0.05\linewidth}|p{0.17\linewidth}|p{0.11\linewidth}|p{0.15\linewidth}|p{0.15\linewidth}|p{0.15\linewidth}|p{0.08\linewidth}}
\hline
\textbf{ID} & \textbf{Description} & \textbf{Req Ref} & \textbf{Input} & \textbf{Expected Output} & \textbf{Actual Output} & \textbf{Result}\\
\hline
C1 & Testing if the relative coordinates (x,y) is written to the CSV file & RDP1, NFR5 & Hand gesture for “a” & Coordinates with identifier “0” (identifier for the letter “a”) are written to the CSV file & Coordinates with identifier “0” were written to the CSV file & Pass\\ \hline
C2 & Testing if the point history coordinates (x,y) is written to the CSV file & RDP1, NFR5 & Hand gesture for “j” & Multiple coordinates with identifier “9” (identifier for the letter “j”) are written to the CSV file & Multiple coordinates with identifier “9” get written to the CSV file & Pass\\ \hline
C3 & Testing to see if a coordinate for each hand joint is written to the CSV file & RDP1 & Hand gesture for “b” & 43 coordinates are written to the CSV file, first the identifier (for the gesture, ie ‘a’, ‘b’, etc.) followed by an x,y coordinate for each joint (21 * 2 + 1 = 43) & 43 coordinates are written to the CSV file every time a gesture is recorded & Pass
\hline
\caption{Tests for Coordinate Export Module}
\end{longtable}

\newpage
\centerline{Tests for Machine Learning Module}

\renewcommand{\arraystretch}{1.2}
\noindent \begin{longtable}{p{0.05\linewidth}|p{0.17\linewidth}|p{0.11\linewidth}|p{0.15\linewidth}|p{0.15\linewidth}|p{0.15\linewidth}|p{0.08\linewidth}}
\hline
\textbf{ID} & \textbf{Description} & \textbf{Req Ref} & \textbf{Input} & \textbf{Expected Output} & \textbf{Actual Output} & \textbf{Result}\\
\hline
D1 & Testing hand detection for similar looking gestures & CFR1, RDP1 & Hand gesture for “m” & m & n & Fail\\ \hline
D2 & Testing hand detection for motion (no input) & CFR1, RDP1 & Static hand gestures (no motions) & no output & z/d & Fail\\ \hline
D3 & Testing hand detection for motion & CFR1, RDP1 & Hand motion for “z” & z & z & Pass\\ \hline
D4 & Testing if gestures that require movement are able to be recognized (motion gestures) & MLFR4, MLFR6, RDP1 & Signing“j” and “z” & j z & j z & Pass\\ \hline
D5 & Test model accuracy by signing different sequences of gestures / introducing variance into the system & MLFR4, NFR1, RDP1 & Sign letters in sequence of a,b,c,d then sign with d, f, z, j & a,b,c,d
d,f,z,j
with 100\% accuracy
 & a,b,c,d
d,f,z,j
 & Pass\\ \hline
D6 & Testing gesture recognition between point history (movement gestures) and keypoint history (static gestures” & MLFR4 & Sign letters in sequence “a”, “b”, “j”, “c”, “z” & a b j c z & j a j z b c z & Fail
\hline
\caption{Tests for Machine Learning Module}
\end{longtable}

\newpage
\centerline{Tests for Training Module}

\renewcommand{\arraystretch}{1.2}
\noindent \begin{longtable}{p{0.05\linewidth}|p{0.17\linewidth}|p{0.11\linewidth}|p{0.15\linewidth}|p{0.15\linewidth}|p{0.15\linewidth}|p{0.08\linewidth}}
\hline
\textbf{ID} & \textbf{Description} & \textbf{Req Ref} & \textbf{Input} & \textbf{Expected Output} & \textbf{Actual Output} & \textbf{Result}\\
\hline
E1 & Mode Selection & N/A & Program is in “Normal Mode”, press number “2” on keyboard & Program goes into “Training Mode” & Program goes into “Training Mode” & Pass\\ \hline
E2 & Test if a .tflite file can be generated from the CSV files & MLFR5, NFR5 & A CSV file with data points from different ASL gestures & A .tflite file that can be used to  recognize the gestures that were recorded & A .tflite file that can be used to  recognize the gestures that were recorded & Pass\\ \hline
E3 & Testing if retraining by adding new data points can change recognition & MLFR7, NFR1, NFR5 & Adding 50 accurate data points to the gesture “Hello” & The accuracy prediction increases & The accuracy prediction decrease from 60\% to 80\% & Pass\\ \hline
E4 & Testing for gesture variation based on user habits through retraining & MLFR7, NFR1, NFR3, NFR7 & Retraining the model with a different method of signing “Hello” & Hello & Hello & Pass
\hline
\caption{Tests for Training Module}
\end{longtable}

\newpage
\centerline{Tests for Text to Speech Module}

\renewcommand{\arraystretch}{1.2}
\noindent \begin{longtable}{p{0.05\linewidth}|p{0.17\linewidth}|p{0.11\linewidth}|p{0.15\linewidth}|p{0.15\linewidth}|p{0.15\linewidth}|p{0.08\linewidth}}
\hline
\textbf{ID} & \textbf{Description} & \textbf{Req Ref} & \textbf{Input} & \textbf{Expected Output} & \textbf{Actual Output} & \textbf{Result}\\
\hline
F1 & Text-to-speech in real-time for individual letters & RDP1, RDP2 & Hand gestures for “a”, “b” and “c”, then hand gesture for “Speak” & Audio output for letters “a”, “b” and “c” & Audio output for letters “a”, “b” and “c” & Pass\\ \hline
F2 & Text-to-speech in real-time for sentence & RDP1, RDP2 & Hand gesture for “I love you”, then hand gesture for “Speak” & Audio output for “I love you” & Audio output for “I love you” & Pass\\ \hline
F3 & Testing hand detection for a series of hand gestures (fast) & MLFR7, CFR1 & A series of hand gestures performed in a very fast speed & Letters for corresponding hand gestures & Some letters are missing & Fail (need to increase fps)\\ \hline
F4 & Test if gesture for “Speak” does not work when in training mode & RDP2, MLFR5 & Program is started, in training mode, and gestures are performed, then gesture “Speak” is performed & No audio output & No audio output & Pass
\hline
\caption{Tests for Text to Speech Module}
\end{longtable}

\newpage
\centerline{Tests for Hardware}

\renewcommand{\arraystretch}{1.2}
\noindent \begin{longtable}{p{0.05\linewidth}|p{0.17\linewidth}|p{0.11\linewidth}|p{0.15\linewidth}|p{0.15\linewidth}|p{0.15\linewidth}|p{0.08\linewidth}}
\hline
\textbf{ID} & \textbf{Description} & \textbf{Req Ref} & \textbf{Input} & \textbf{Expected Output} & \textbf{Actual Output} & \textbf{Result}\\
\hline
G1 & Camera is set up on the Raspberry Pi & CFR1 & Raspistill command to take a picture & A picture & A picture & Pass\\ \hline
G2 & Test if the Raspberry Pi can capture the input from the camera and translate ASL in real time & ??? & Program is started on the Raspberry Pi & The Raspberry Pi should be able to use the camera to detect and translate ASL in real time & The Raspberry Pi camera does not display the video with an adequate frame rate, making translation undoable & Fail\\ \hline
G3 & Real-time video is captured and displayed on screen & CFR1, CFR2 & Views in front of the camera & Views in front of the camera are displayed & Views in front of the camera are displayed & Pass
\hline
\caption{Tests for Hardware}
\end{longtable}

\newpage
\centerline{Text and String Display}

\renewcommand{\arraystretch}{1.2}
\noindent \begin{longtable}{p{0.05\linewidth}|p{0.17\linewidth}|p{0.11\linewidth}|p{0.15\linewidth}|p{0.15\linewidth}|p{0.15\linewidth}|p{0.08\linewidth}}
\hline
\textbf{ID} & \textbf{Description} & \textbf{Req Ref} & \textbf{Input} & \textbf{Expected Output} & \textbf{Actual Output} & \textbf{Result}\\
\hline
H1 & Real-time text display for hand gestures (normal speed) & RDP1 & hand gestures for “d” and “a” performed in a reasonable speed & Output the corresponding letters “d” and “a” besides user’s hand & Output the corresponding letters “d” and “a” besides user’s hand & Pass\\ \hline
H2 & Real-time text display for hand gestures (super fast) & RDP1 & hand gestures performed in a super fast speed & Letters for corresponding hand gestures & Some letters output are missing & Fail (need to increase fps)\\ \hline
H3 & String display for one hand gesture & MLFR6, MLFR4, NFR1 & hand gestures for “d” & “d” is displayed as string at the bottom of the screen & “d” is displayed as string at the bottom of the screen & Pass\\ \hline
H4 & String display for a series of hand gestures (slow speed) & MLFR6, MLFR4, NFR1 & hand gestures for “d” and “a” and “I love you” with a pause of 4 seconds & “d a I love you” is displayed as string at the bottom of the screen & “d d a a I love you I love you” is displayed as string at the bottom of the screen & Fail\\ \hline
H5 & String display for a series of hand gestures (normal speed) & MLFR6, MLFR4, NFR1 & hand gestures for “d” and “a” and “I love you” with a pause of 1 to 2 seconds & “d a I love you” is displayed as string at the bottom of the screen & “d a I love you” is displayed as string at the bottom of the screen
& Pass\\ \hline
H6 & String display for a series of hand gestures (fast speed) & MLFR6, MLFR4, NFR1 & hand gestures for “d” and “a” and “I love you” without pause & “d a I love you” is displayed as string at the bottom of the screen & “d I love you” is displayed as string at the bottom of the screen & Fail\\ \hline
H7 & Modifying string display & N/A & Pressing “Backspace” or “Space” & “Backspace” deletes a character in string, “Space” adds a space in string & “Backspace” deletes a character in string, “Space” adds a space in string & Pass\\ \hline
H8 & String display is cleared after audio output & RDP1, RDP2 & Hand gestures are performed, and then perform hand gesture for “Speak” & Current string is cleared & Current string is cleared & Pass\\ \hline
H9 & Test if gestures are not written to string when in training mode & N/A & Program is started, in training mode, and gestures are being performed & Nothing is being added to the string and nothing is displayed at the bottom & Nothing is added to the string and nothing is displayed at the bottom & Pass
\hline
\caption{Text and String Display}
\end{longtable}

\bibliographystyle{plainnat}
\bibliography{../../refs/References}

\newpage{}
\section*{Appendix --- Reflection}

The information in this section will be used to evaluate the team members on the
graduate attribute of Reflection.  Please answer the following question:

\begin{enumerate}
  \item In what ways was the Verification and Validation (VnV) Plan different
  from the activities that were actually conducted for VnV?  If there were
  differences, what changes required the modification in the plan?  Why did
  these changes occur?  Would you be able to anticipate these changes in future
  projects?  If there weren't any differences, how was your team able to clearly
  predict a feasible amount of effort and the right tasks needed to build the
  evidence that demonstrates the required quality?  (It is expected that most
  teams will have had to deviate from their original VnV Plan.)
\end{enumerate}

\end{document}