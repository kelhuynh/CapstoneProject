\documentclass[12pt, titlepage]{article}

\usepackage{amsmath, mathtools}

\usepackage[round]{natbib}
\usepackage{amsfonts}
\usepackage{amssymb}
\usepackage{graphicx}
\usepackage{colortbl}
\usepackage{xr}
\usepackage{hyperref}
\usepackage{longtable}
\usepackage{xfrac}
\usepackage{tabularx}
\usepackage{float}
\usepackage{siunitx}
\usepackage{booktabs}
\usepackage{multirow}
\usepackage[section]{placeins}
\usepackage{caption}
\usepackage{fullpage}

\hypersetup{
bookmarks=true,     % show bookmarks bar?
colorlinks=true,       % false: boxed links; true: colored links
linkcolor=red,          % color of internal links (change box color with linkbordercolor)
citecolor=blue,      % color of links to bibliography
filecolor=magenta,  % color of file links
urlcolor=cyan          % color of external links
}

\usepackage{array}

\externaldocument{../../SRS/SRS}

%% Comments

\usepackage{color}

\newif\ifcomments\commentstrue %displays comments
%\newif\ifcomments\commentsfalse %so that comments do not display

\ifcomments
\newcommand{\authornote}[3]{\textcolor{#1}{[#3 ---#2]}}
\newcommand{\todo}[1]{\textcolor{red}{[TODO: #1]}}
\else
\newcommand{\authornote}[3]{}
\newcommand{\todo}[1]{}
\fi

\newcommand{\wss}[1]{\authornote{blue}{SS}{#1}} 
\newcommand{\plt}[1]{\authornote{magenta}{TPLT}{#1}} %For explanation of the template
\newcommand{\an}[1]{\authornote{cyan}{Author}{#1}}

%% Common Parts

\newcommand{\progname}{Mechatronics} % PUT YOUR PROGRAM NAME HERE
\newcommand{\authname}{Team \#20, Team Name
\\ Robert Zhu zhul49
\\ Zifan Meng mengz17
\\ Jiahui Chen chenj194
\\ Kelvin Huynh huynhk12
\\ Runze Zhu zhur25
\\ Mirza Nafi Hasan hasanm21} % AUTHOR NAMES                  

\usepackage{hyperref}
    \hypersetup{colorlinks=true, linkcolor=blue, citecolor=blue, filecolor=blue,
                urlcolor=blue, unicode=false}
    \urlstyle{same}
                                


\begin{document}

\title{Module Interface Specification for \progname{}} 
\author{\authname}
\date{\today}

\maketitle

\pagenumbering{roman}

\section{Revision History}

\begin{tabularx}{\textwidth}{p{3cm}p{2cm}X}
\toprule {\bf Date} & {\bf Version} & {\bf Notes}\\
\midrule
January 18, 2023 & 1.0 & Everyone - Initial MIS Draft\\
\bottomrule
\end{tabularx}

~\newpage

\section{Symbols, Abbreviations and Acronyms}

See SRS Documentation at \url{https://github.com/kelhuynh/OpenASL/blob/main/docs/SRS/SRS.pdf}

\newpage

\tableofcontents

\newpage

\pagenumbering{arabic}

\section{Introduction}

The following document details the Module Interface Specifications for
OpenASL, a device developed with the aim of translating sign language into text-to-speech, 
with the purpose of helping members of the deaf and mute community communicate with those 
who do not know sign language.\\

Complementary documents include the System Requirement Specifications
and Module Guide.  The full documentation and implementation can be
found at \url{https://github.com/kelhuynh/OpenASL/}.\\

\section{Notation}

N/A\\

\section{Module Decomposition}

The following table is taken directly from the Module Guide document for this project.

\begin{table}[h!]
\centering
\begin{tabular}{p{0.3\textwidth} p{0.6\textwidth}}
\toprule
\textbf{Level 1} & \textbf{Level 2}\\
\midrule

{Hardware-Hiding Module} & Video Capture Module \\
\midrule

\multirow{7}{0.3\textwidth}{Behaviour-Hiding Module} & Text-to-Speech Module\\
& Key Point Classification Module - Communicates with ML module with data from coordinate normalization module\\
& Training Module - Communicates with ML module to update dataset\\
& Coordinate Export Module - Read data from video capture and stores into file\\
& Motion Tracking Module - Controller (ties everything together)\\

\midrule

\multirow{3}{0.3\textwidth}{Software Decision Module} & Video Analysis Module - requires data to be used\\
& Machine Learning Module\\
& Coordinate Normalization Module\\
\bottomrule


\end{tabular}
\caption{Module Hierarchy}
\label{TblMH}
\end{table}

\section{Connection Between Requirements and Design} \label{SecConnection}

The design of the system is intended to satisfy the requirements developed in
the SRS. In this stage, the system is decomposed into modules. The connection
between requirements and modules is listed in Table \ref{TblRT}.

\begin{table}[H]
  \centering
  \begin{tabular}{p{0.2\textwidth} p{0.6\textwidth}}
  \toprule
  \textbf{Req.} & \textbf{Modules}\\
  \midrule
  CFR1 & \mref{M1}, \mref{M4}\\
  CFR2 & \mref{M5}\\
  MLFR1 & \mref{M2}\\
  MLFR2 & \mref{M3}\\
  MLFR3 & \mref{M4}\\
  MLFR4 & \mref{M6}, \mref{M9}, \mref{M10}\\
  MLFR5 & \mref{M4}\\
  MLFR6 & \mref{M4}\\
  MLFR7 & \mref{M7}, \mref{M8}\\
  NFR1 & \mref{M5}, \mref{M6}, \mref{M7}\\
  NFR2 & \mref{M1}, \mref{M4}\\
  NFR3 & \mref{M9}\\
  NFR4 & \mref{M9}\\
  NFR5 & \mref{M7}\\
  NFR6 & \mref{M10}\\
  NFR7 & \mref{M1}, \mref{M4}, \mref{M9}\\
  
  \bottomrule
  \end{tabular}
  \caption{Trace Between Requirements and Modules}
  \label{TblRT}
  \end{table}


\newpage

\section{MIS of Motion Tracking Module (M1)} \label{m1}

\subsection{Module}

motionTrack

\subsection{Uses}

Video Capture, Coordinate Normalization, Coordinate Export, Video Analysis, Keypoint Classification, TTS

\subsection{Syntax}

\subsubsection{Exported Constants}

\begin{center}
\begin{tabular}{p{5cm} p{4cm} p{4cm} p{2cm}}
\hline
\textbf{Name} & \textbf{In} & \textbf{Out} & \textbf{Exceptions} \\
\hline
results & image & Object & - \\
hand\_landmarks & - & Tuple of tuples & - \\
handedness & - & \mathbb{R} & - \\
\hline
\end{tabular}
\end{center}

\subsection{Semantics}

\subsubsection{State Variables}

None

\subsubsection{Environment Variables}

f - file variable for coordinate export purposes

\subsubsection{Assumptions}

None

\subsubsection{Access Routine Semantics}

\noindent motionTrack():
\begin{itemize}
\item output: video consisting of overlay for hand gesture classification into ASL\\
\item exception: exc := cv2.error
\end{itemize}

~\newpage

\section{MIS of Coordinate Normalization Module (M2)} \label{m2}

\subsection{Module}

Coordinate Normalization

\subsection{Uses}

Video Capture

\subsection{Syntax}

\subsubsection{Exported Constants}

\begin{center}
\begin{tabular}{p{5cm} p{4cm} p{4cm} p{2cm}}
\hline
\textbf{Name} & \textbf{In} & \textbf{Out} & \textbf{Exceptions} \\
\hline
pre\_processed\_landmark\_list & landmark\_list & Tuple of tuples & - \\
\hline
\end{tabular}
\end{center}

\subsection{Semantics}

\subsubsection{State Variables}

None

\subsubsection{Environment Variables}

None

\subsubsection{Assumptions}

None

\subsubsection{Access Routine Semantics}

\noindent pre\_process\_landmark(landmark\_list):
\begin{itemize}
\item output: tuple of 20 tuples consisting of x and y coordinates for each hand joint
\item exception: exc := ListIndexOutofBounds
\end{itemize}

\subsubsection{Local Functions}
\begin{itemize}
\item \_\_calc\_landmark\_list
\end{itemize}

~\newpage

\section{MIS of Coordinate Export Module (M3)} \label{m3}

\subsection{Module}

Coordinate Export

\subsection{Uses}

Coordinate Normalization

\subsection{Syntax}

\subsubsection{Exported Constants}

\begin{center}
\begin{tabular}{p{5cm} p{4cm} p{4cm} p{2cm}}
\hline
\textbf{Name} & \textbf{In} & \textbf{Out} & \textbf{Exceptions} \\
\hline
keypoint.csv & - & File containing normalized coordinates & - \\
\hline
\end{tabular}
\end{center}

\subsection{Semantics}

\subsubsection{State Variables}

None

\subsubsection{Environment Variables}

None

\subsubsection{Assumptions}

None

\subsubsection{Local Functions}
\begin{itemize}
\item \_\_make\_csv
\end{itemize}

~\newpage

\section{MIS of Video Capture Module (M4)} \label{m4}

\subsection{Module}

Video Capture

\subsection{Uses}

None

\subsection{Syntax}

\subsubsection{Exported Constants}

\begin{center}
\begin{tabular}{p{5cm} p{4cm} p{4cm} p{2cm}}
\hline
\textbf{Name} & \textbf{In} & \textbf{Out} & \textbf{Exceptions} \\
\hline
success & - & \mathbb{R} & - \\
image & - & Object & - \\
\hline
\end{tabular}
\end{center}

\subsection{Semantics}

\subsubsection{State Variables}

None

\subsubsection{Environment Variables}

success - indicates that the camera input is initialized for use

\subsubsection{Assumptions}

There is an available camera connected to the system

~\newpage

\section{MIS of Video Analysis Module (M5)} \label{m5}

\subsection{Module}

Video Analysis

\subsection{Uses}

Coordinate Normalization, Video Capture

\subsection{Syntax}

\subsubsection{Exported Access Programs}

\begin{center}
\begin{tabular}{p{5cm} p{4cm} p{4cm} p{2cm}}
\hline
\textbf{Name} & \textbf{In} & \textbf{Out} & \textbf{Exceptions} \\
\hline
draw\_bounding\_rect & - & image & - \\
draw\_landmarks & - & image & - \\
draw\_info\_text & - & image & - \\
\hline
\end{tabular}
\end{center}

\subsection{Semantics}

\subsubsection{State Variables}

None

\subsubsection{Environment Variables}

None

\subsubsection{Assumptions}

None

\subsubsection{Access Routine Semantics}

\noindent draw\_bounding\_rect(self, use\_brect, image, brect):
\begin{itemize}
\item output: image with overlaid bounding rectangle around hand
\item exception: exc := None
\end{itemize}

\noindent draw\_landmarks(self, image, landmark\_point):
\begin{itemize}
\item output: image with overlaid hand joints and connections
\item exception: exc := None
\end{itemize}

\noindent draw\_info\_text(self, image, brect, handedness, hand\_sign\_text):
\begin{itemize}
\item output:  image with overlaid classifier label
\item exception: exc := None
\end{itemize}

\subsubsection{Local Functions}
\begin{itemize}
\item \_\_calc\_bounding\_rect
\end{itemize}

~\newpage

\section{MIS of Keypoint Classification Module (M6)} \label{M6}

\subsection{Module}

Keypoint Classification\\

\subsection{Uses}

Coordinate Export, Machine Learning\\

\subsection{Syntax}

\subsubsection{Exported Constants}

\begin{center}
\begin{tabular}{p{3cm} p{3cm} p{2cm} p{4cm}}
\hline
\textbf{Name} & \textbf{In} & \textbf{Out} & \textbf{Exceptions} \\
\hline
result_index & landmark_list & R & ListIndexOutofRange \\
\hline
\end{tabular}
\end{center}

\subsection{Semantics}

\subsubsection{State Variables}

None\\

\subsubsection{Environment Variables}

None\\

\subsubsection{Assumptions}

None\\

\subsubsection{Local Functions}
\begin{itemize}
\item \_\_call\_\_
\end{itemize}

~\newpage

\section{MIS of Machine Learning Module (M7)} \label{M7}

\subsection{Module}

Machine Learning\\

\subsection{Uses}

Keypoint Classification, Coordinate Export, Coordinate Normalization, Training\\

\subsection{Syntax}

\subsubsection{Exported Constants}

None\\

\subsection{Semantics}

\subsubsection{State Variables}

None\\

\subsubsection{Environment Variables}

None\\

\subsubsection{Assumptions}

None\\

~\newpage

\section{MIS of Training Module (M8)} \label{M8}

\subsection{Module}

ML Train\\

\subsection{Uses}

Coordinate Export\\

\subsection{Syntax}

\subsubsection{Exported Constants}

\begin{center}
\begin{tabular}{p{5cm} p{1cm} p{4cm} p{2cm}}
\hline
\textbf{Name} & \textbf{In} & \textbf{Out} & \textbf{Exceptions} \\
\hline
keypoint_classifier.hdf5 & - & Hierarchical Data Format file & - \\
\hline
keypoint_classifier.tflite & - & ML model & - \\
\hline
\end{tabular}
\end{center}

\subsection{Semantics}

\subsubsection{State Variables}

None\\

\subsubsection{Environment Variables}

None\\

\subsubsection{Assumptions}

None\\

~\newpage

\section{MIS of Text to Speech Module (M9)} \label{M9}

\subsection{Module}

TTS\\

\subsection{Uses}

Video Analysis, Keypoint Classification, Coordinate Normalization\\

\subsection{Notes}

To be implemented\\

~\newpage

\section{MIS of Hardware Hiding Module (M10)} \label{M10}

\subsection{Module}

Hardware hiding\\

\subsection{Notes}

To be implemented\\

\newpage

\bibliographystyle {plainnat}
\bibliography {../../../refs/References}

\end{document}