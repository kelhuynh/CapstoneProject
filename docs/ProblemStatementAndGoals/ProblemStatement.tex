\documentclass{article}

\usepackage{tabularx}
\usepackage{booktabs}

\title{Problem Statement and Goals\\\progname}

\author{\authname}

\date{}

%% Comments

\usepackage{color}

\newif\ifcomments\commentstrue %displays comments
%\newif\ifcomments\commentsfalse %so that comments do not display

\ifcomments
\newcommand{\authornote}[3]{\textcolor{#1}{[#3 ---#2]}}
\newcommand{\todo}[1]{\textcolor{red}{[TODO: #1]}}
\else
\newcommand{\authornote}[3]{}
\newcommand{\todo}[1]{}
\fi

\newcommand{\wss}[1]{\authornote{blue}{SS}{#1}} 
\newcommand{\plt}[1]{\authornote{magenta}{TPLT}{#1}} %For explanation of the template
\newcommand{\an}[1]{\authornote{cyan}{Author}{#1}}

%% Common Parts

\newcommand{\progname}{Mechatronics} % PUT YOUR PROGRAM NAME HERE
\newcommand{\authname}{Team \#20, Team Name
\\ Robert Zhu zhul49
\\ Zifan Meng mengz17
\\ Jiahui Chen chenj194
\\ Kelvin Huynh huynhk12
\\ Runze Zhu zhur25
\\ Mirza Nafi Hasan hasanm21} % AUTHOR NAMES                  

\usepackage{hyperref}
    \hypersetup{colorlinks=true, linkcolor=blue, citecolor=blue, filecolor=blue,
                urlcolor=blue, unicode=false}
    \urlstyle{same}
                                


\begin{document}

\maketitle

\begin{table}[hp]
\caption{Revision History} \label{TblRevisionHistory}
\begin{tabularx}{\textwidth}{llX}
\toprule
\textbf{Date} & \textbf{Developer(s)} & \textbf{Change}\\
\midrule
9/26/2022 & Everyone & Initial Revision\\
... & ... & ...\\
\bottomrule
\end{tabularx}
\end{table}

\section{Problem Statement}

\subsection{Problem}
Communication is a key component of both the professional workplace and personal life. Many difficulties
can occur especially when the individual has a disability that prevents them from being able to speak properly.
Sign language has been a method to help bridge those who are deaf/mute with other people, but as with language,
both parties are required to understand it to clearly communicate with each other. The Sign Language Translator
is a device to help further close that gap by introducing a sensor that can translate hand motions and gestures
from the American Sign Language (ASL) into a text to speech application on their phones. This method will be able
to provide real-time instant feedback to simulate a spoken conversation and can eliminate the need for a third party
translator as the individual is able to express themselves freely, thus improving their quality of life in society. 

\subsection{Inputs and Outputs}

The input of the sign language translator should be hand gestures demonstrated by people, the device is capable of 
converting hand gestures to words, phrases, or sentences in designated languages. The process of input to output 
should be as follows:

\begin{itemize}
    \item An individual should be able to use sign language as usual
    \item A camera should capture those hand signals for the system
    \item The system should be able to generate an text-to-speech output 
    based upon the initial signing
\end{itemize}

\subsection{Stakeholders}

The stakeholders for our project are people who have hearing problems and need to use sign language for their daily
communication. This can also include various accessibility services for various companies, whether that be in education
or entertainment. Our project can benefit anyone or anything that requires a sign language interpreter.

\subsection{Environment}

\wss{Hardware and software}

\section{Goals}

Reliable and Accurate Translations:\\
The Sign Language Translator requires extensive training on the sensors to capture precise hand motion and ignore any
human error on the user\textquotesingle s part. The processing unit should be able to identify each letter within the American Sign Language
using the data collected and transmit dialogue accurately to the user\textquotesingle s request.\\

~\\Real Time Translations:\\
User\textquotesingle s should never be required to wait an extensive period of time for the device to process their hand motion and provide
a translation. The Sign Language Translator should simulate a real time conversation between regular people to deliver a seamless
transition for other parties during presentations or social interactions.\\

~\\Ease of Use:\\
The user experience is crucial for a communication device. The Sign Language Translator should require minimal time 
and effort to set up. Once set up, the device should not require much maintenance or updates. Most importantly,
the device should not hinder the user\textquotesingle s ability to perform the gestures and hand motions of sign language.\\

~\\Affordability:\\
The Sign Language Translator should be affordable for the end users as to reduce the need of requiring an actual translator 
to accompany the user during their tasks. The device should remain functional whenever it is required to be used, and the hardware components
 of the device should be simple and cost-effective.\\

~\\Customizable to User:\\
As with language, different people might have a certain way of pronouncing a phrase or word and likewise the same could be said with Sign Language 
with slightly different gestures. The device should be able to adapt to the user and recognize the unique motions instead of forcing the user to 
slow down for the device. 

\section{Stretch Goals}

Portable:\\
The final device, while requiring OpenCV to scan and process hand motion, should become more portable and lightweight for the user
to move around, so as to not interfere with the user\textquotesingle s regular activities. The translator text to speech should
become an application on all phone brands as for any user with the required equipment to be able to begin using.\\

~\\Expanding to Different Languages:\\
As a universal sign language does not exist at the moment, there exists dead/mute individuals who use another form of sign language
other than the American Sign language. These include the British, Australian and New Zealand Sign Language (BANZSL), the Chinese Sign
Language (CSL), Arabic Sign language, and much more. The device should be able to understand and translate these new hand motions and
generate a translation in their native language for this product to be used on a global scale.\\

~\\Sign Language Education:\\
The final device should be able to recognize the different hand motions and gestures of sign language in order to accurately 
translate them. This would make the device an excellent educational tool for those looking to learn sign language. The device could 
provide feedback and tell users how to improve their gestures using it\textquotesingle s accurate hand tracking to help teach those unfamiliar with 
sign language.

\end{document}
