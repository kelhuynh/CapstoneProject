\documentclass{article}

\usepackage{booktabs}
\usepackage{tabularx}
\usepackage{multirow}
\usepackage{float}
\usepackage{longtable}
\usepackage[none]{hyphenat}
\usepackage[document]{ragged2e}
\usepackage{fullpage}
\usepackage{tabu}
\usepackage{enumitem}

%% Comments

\usepackage{color}

\newif\ifcomments\commentstrue %displays comments
%\newif\ifcomments\commentsfalse %so that comments do not display

\ifcomments
\newcommand{\authornote}[3]{\textcolor{#1}{[#3 ---#2]}}
\newcommand{\todo}[1]{\textcolor{red}{[TODO: #1]}}
\else
\newcommand{\authornote}[3]{}
\newcommand{\todo}[1]{}
\fi

\newcommand{\wss}[1]{\authornote{blue}{SS}{#1}} 
\newcommand{\plt}[1]{\authornote{magenta}{TPLT}{#1}} %For explanation of the template
\newcommand{\an}[1]{\authornote{cyan}{Author}{#1}}

%% Common Parts

\newcommand{\progname}{Mechatronics} % PUT YOUR PROGRAM NAME HERE
\newcommand{\authname}{Team \#20, Team Name
\\ Robert Zhu zhul49
\\ Zifan Meng mengz17
\\ Jiahui Chen chenj194
\\ Kelvin Huynh huynhk12
\\ Runze Zhu zhur25
\\ Mirza Nafi Hasan hasanm21} % AUTHOR NAMES                  

\usepackage{hyperref}
    \hypersetup{colorlinks=true, linkcolor=blue, citecolor=blue, filecolor=blue,
                urlcolor=blue, unicode=false}
    \urlstyle{same}
                                



\title{Hazard Analysis\\\progname}
\author{\authname}

\date{October 19, 2022}

\begin{document}

\maketitle

~\newpage

\tableofcontents
\listoftables

~\newpage

\section*{Revision History}
\begin{table}[hp]
\caption{Revision History} \label{TblRevisionHistory}
\begin{tabularx}{\textwidth}{llX}
\toprule
\textbf{Date} & \textbf{Developer(s)} & \textbf{Change}\\
\midrule
Oct 19, 2022 & Everyone & Initial version\\

... & ... & ...\\
\bottomrule
\end{tabularx}
\end{table}

\newpage
\textheight 9in

\section{Introduction}

This document is a hazard analysis of Group 20\textquotesingle s ASL Translator. The ASL Translator is a real-time sign language translation device intended to aid 
individuals who are hard of hearing in day to day communication tasks. This device may also be used for the purpose of facilitating the learning of 
sign language in an educational setting.\\

\section{Scope and Purpose}

The purpose of this document is to identify hazards that may occur when using the ASL Translator specifically in the components, their causes and consequences on user 
operation, hazard mitigation, and their respective requirements. The scope of this document will span from the time that the user performs their sign language to the
time that the inputted sign is translated. This will include motion tracking, ML processing \\

\section{System Boundaries}

The system boundaries for the hazard analysis includes the Raspberry Pi, of which the software will be installed in, and the components of the software itself. These components
are listed below:

\begin{enumerate}
    \item Video Capture using a Camera
\end{enumerate}

\hspace*{9mm}This component enables users' sign language to be captured in real-time.

\begin{enumerate}[resume]
    \item Motion Tracking
\end{enumerate}

\hspace*{9mm}\hangindent=9mm This component processes the hand joint coordinates retrieved from the video capture component into data that can be used by the machine learning model.

\begin{enumerate}[resume]
    \item Machine Learning Model for ASL Processing
\end{enumerate}

\hspace*{9mm}\hangindent=9mm This component analyzes the coordinate data and identifies the sign being displayed through comparison to the data it was trained with during supervised learning.

\begin{enumerate}[resume]
    \item Text to Speech
\end{enumerate}

\hspace*{9mm}\hangindent=9mm This component utilizes the sign identification from the machine learning model to playback the translated gesture through a speaker.

\section{Definition of Hazard}

The definition of a hazard is based on the definition from Nancy Leveson\textquotesingle s work. A hazard is a property or condition in the system along with a condition in the
environment that results in a loss. A hazard is anything that can cause our system to function incorrectly, or not function at all. In the ASL Translator, there
exists only hazards that affect the proper and intended operation of the device.\\


\section{Critical Assumptions}
At this time, there are no critical assumptions to be made.

\section{Failure Modes and Effects Analysis}

The hazard analysis tool being used is the Failure Modes and Effects Analysis (FMEA). This will enable hazard identification and analysis such that additional safety requirements can 
be created and considered in the implementation of the project.\\

\subsection{Hazards Out of Scope}

The out of scope hazards for our project is primarily based on the user\textquotesingle s decision. This is because we do not have control over the following:\\
\begin{itemize}
    \item The camera that is to be used in conjunction with the system
    \item The Raspberry Pi model and microSD card capacity being used
\end{itemize}

Both components listed above are essential to the functionality of the system. However, there is no enforcement on these aspects as the user may prefer something 
less costly or more costly. The user\textquotesingle s decision towards these components may vary and can affect the overall performance of the system. Steps will be taken to minimize 
the impact of the user\textquotesingle s choice in these categories such as ensuring backwards compatibility is possible with our implementation of code on the Raspberry Pi and camera 
calibration methods.\\

\subsection{Failure Modes and Effects Analysis Table}

Below is the FMEA table for the project.

\tabulinesep=1.2mm
\hspace*{-1cm}\begin{longtabu} to \textwidth {|p{0.1\linewidth}|p{0.125\linewidth}|p{0.125\linewidth}|p{0.125\linewidth}|p{0.255\linewidth}|p{0.05\linewidth}|p{0.05\linewidth}|}
\caption{Failure Modes and Effects Analysis}\label{TblFMEA}\\
\firsthline
\multicolumn{1}{|c|}{Component} & \multicolumn{1}{c|}{Failure Modes}          & \multicolumn{1}{c|}{Effects of Failure}          & \multicolumn{1}{c|}{Causes of Failure} & \multicolumn{1}{c|}{Recommended Action} & \multicolumn{1}{c|}{Req.} & \multicolumn{1}{c|}{Ref.} \\
\hline \endfirsthead
\hline
\multicolumn{1}{|c|}{Component} & \multicolumn{1}{c|}{Failure Modes}          & \multicolumn{1}{c|}{Effects of Failure}          & \multicolumn{1}{c|}{Causes of Failure} & \multicolumn{1}{c|}{Recommended Action} & \multicolumn{1}{c|}{Req.} & \multicolumn{1}{c|}{Ref.} \\
\hline \endhead
\hline
\multicolumn{7}{r}{Continued on next page} \\
\hline \endfoot
\hline \endlastfoot

\multirow[t]{1}{=}{Raspberry Pi}& Fail to output translated results           & The output is not properly displayed             & a. Code execution failure due to improper flashing to board & a. Reboot the board and ensure that the correct software and dependencies are loaded & a. SR2 & H1-1 \\ \cline{2-7}
                                & \multirow[t]{1}{=}{Hardware failure (board)}& \multirow[t]{1}{=}{Raspberry Pi cannot function} & a. The board is not powered due to faulty power supply & a. Ensure that the board is properly plugged in. Use another power cable to verify that the board itself is not faulty. The raspberry pi is equipped with a polyfuse to prevent over-current. If the board does not power on after 24 hours, the fuse should be replaced. & a. HR1 & \multirow[t]{3}{=}{H1-2}\\ \cline{4-6}
\multirow[t]{5}{=}{Raspberry Pi}& \multirow[t]{3}{=}{Hardware failure (board)}& \multirow[t]{3}{=}{Raspberry Pi cannot function} & b. The software is corrupted & b. Remove the microSD card and ensure that the card is not corrupted using a computer. If it is corrupted, attempt to reformat the microSD card and write the software onto it again. & b. HR1 & \multirow[t]{3}{=}{H1-2}\\ \cline{4-6}
                                &                                             &                                                  & c. The board is faulty or defective & c. Attempt to run the software on an identical model using the same cables, microSD card, and peripherals. If the software works, then the existing board is faulty and should be replaced. Otherwise, test using the newer components until the problem is isolated. & c. HR1 & \\ \cline{4-6}
                                &                                             &                                                  & d. The microSD card cannot be read & d. Same as H1-2b & d. HR1 & \\ \cline{2-7}
                                & \multirow[t]{1}{=}{Power supply failure}    & \multirow[t]{1}{=}{Risk of electric shock}       & a. Faulty or damaged power supply & a. Replace power supply as soon as possible & a. HR3 & \multirow[t]{1}{=}{H1-3}\\ \hline
\multirow[t]{3}{=}{Camera}      & \multirow[t]{2}{=}{Failing to detect motion}& \multirow[t]{3}{=}{Device is unable to detect or translate sign language} & a. Poor lighting conditions & a. Ensure the lighting conditions in the working environment is sufficient. This will enable the camera to get sufficient light exposure to recognize hand motions. & a. ER1 & \multirow[t]{1}{=}{H2-1}\\ \cline{4-6}
                                &                                             &                                                  & b. Unclear hand gestures & b. Move hands out of frame then back in to see if hand joints are being detected properly then sign. If problem persists refer to H2-2a or replace camera &            & \\ \cline{2-2} \cline{4-7}
                                & \multirow[t]{1}{=}{Hardware failure}        &                                                  & a. Issues with the lenses & a. Ensure that the camera lenses are not clouded by debris and recalibrate the camera. If the problem persists, replace the lenses. & a. HR2 & \multirow[t]{1}{=}{H2-2}\\\cline{4-6}
\multirow[t]{1}{=}{Camera}      & \multirow[t]{1}{=}{Hardware failure}        & \multirow[t]{1}{=}{Device is unable to detect or translate sign language} & b. Electrical power issues & b. Make sure that the camera is properly plugged into the system. Test the camera on another device such as a PC, and if the camera is functional. Test the power supply of the board. Refer to H1-2a for power supply testing. & b. HR2 & \multirow[t]{1}{=}{H2-2}\\ \hline
\multirow[t]{4}{=}{The machine learning model for ASL} & \multirow[t]{4}{=}{ML model is not accurate} & \multirow[t]{4}{=}{The device will produce incorrect or unexpected outputs} & a. Incorrect data set is being used & a. Double check to see if the input data is correct when training the model & a. MLR1 & \multirow[t]{4}{=}{H3-1}\\ \cline{4-6}
                                &                                             &                                                  & b. ML model is misusing data and turning it into meaningless input & b. Pass irrelevant input through to see if the error is consistent with input that is correct. This indicates that there is an error with how the model is programmed & b. MLR1 & \\ \cline{4-6}
                                &                                             &                                                  & c. The training dataset is not randomized enough resulting in the model anticipating specific orders of input & c. Shuffle the data during training to introduce variance into the model to boost output accuracy & c. MLR1 & \\ \cline{4-6}
                                &                                             &                                                  & d. Model is underfit or overfit resulting in poor generalization outside of the training dataset & d. Same as H3-1c & d. MLR2 & \\ \hline
\multirow[t]{2}{=}{Motion Tracking} & \multirow[t]{2}{=}{Script is not running} & \multirow[t]{2}{=}{The device will not be able to translate sign language} & a. Code execution error & a. Ensure script is error-free and has error handling for foreseeable caes of error & a. SR1 & \multirow[t]{2}{=}{H4-1}\\ \cline{4-6}
                                &                                             &                                                  & b. Code was not flashed onto the board correctly & b. Same as H1-1a & b. SR2 & \\ \hline
\end{longtabu}\hspace*{-1cm}

\section{Requirements}
\subsection{Hardware Requirement}
HR1: The Raspberry Pi board should be checked and tested before using it in the project.\\
HR2: The camera should be checked before each use to ensure the lenses are clean and working properly.\\
HR3: All electrical components of the system should be checked before use to ensure that there is no damage\\
\subsection{Environment Requirement}
ER1: Ensure there is adequate lighting so the camera can detect and translate the hand motions.
\subsection{Machine Learning Requirement}
MLR1: Dataset should be properly chosen and verified, sufficient amount of data should be included in the dataset. Preprocessing should be done on the dataset for standardization and normalization.\\
MLR2: ML model should be properly chosen and tested, parameters optimization should be done for balanced results and optimized performance.
\subsection{Script Requirement}
SR1: Various software testing (Black box testing, White box testing, etc) should be performed on the script before it is flashed onto the board.\\
SR2: Complete functionality testing should be performed after the code is flashed onto the board.

\section{Roadmap}
The hazard analysis has resulted in new safety requirements that were not considered in the initial revision of the 
System Requirements Specification (SRS). Given the current time and budget constraints, while some hazards will be accounted 
for and removed, some hazards might still remain in the final revision of the project. In those cases, notices will be provided 
to the user to ensure safety and proper regulation. Throughout the next steps of the project, the hazard analysis will continuously 
be consulted to determine risks that have already been considered in the implementation and risks that will be removed in future updates.

\end{document}
