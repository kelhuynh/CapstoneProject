\documentclass{article}

\usepackage{booktabs}
\usepackage{tabularx}
\usepackage{multirow}
\usepackage{float}
\usepackage{longtable}
\usepackage[none]{hyphenat}
\usepackage[document]{ragged2e}
\usepackage{fullpage}
\usepackage{tabu}

%% Comments

\usepackage{color}

\newif\ifcomments\commentstrue %displays comments
%\newif\ifcomments\commentsfalse %so that comments do not display

\ifcomments
\newcommand{\authornote}[3]{\textcolor{#1}{[#3 ---#2]}}
\newcommand{\todo}[1]{\textcolor{red}{[TODO: #1]}}
\else
\newcommand{\authornote}[3]{}
\newcommand{\todo}[1]{}
\fi

\newcommand{\wss}[1]{\authornote{blue}{SS}{#1}} 
\newcommand{\plt}[1]{\authornote{magenta}{TPLT}{#1}} %For explanation of the template
\newcommand{\an}[1]{\authornote{cyan}{Author}{#1}}

%% Common Parts

\newcommand{\progname}{Mechatronics} % PUT YOUR PROGRAM NAME HERE
\newcommand{\authname}{Team \#20, Team Name
\\ Robert Zhu zhul49
\\ Zifan Meng mengz17
\\ Jiahui Chen chenj194
\\ Kelvin Huynh huynhk12
\\ Runze Zhu zhur25
\\ Mirza Nafi Hasan hasanm21} % AUTHOR NAMES                  

\usepackage{hyperref}
    \hypersetup{colorlinks=true, linkcolor=blue, citecolor=blue, filecolor=blue,
                urlcolor=blue, unicode=false}
    \urlstyle{same}
                                


\title{Hazard Analysis\\\progname}
\author{\authname}

\date{October 19, 2022}

\begin{document}

\maketitle

~\newpage

\tableofcontents
\listoftables

~\newpage

\section*{Revision History}
\begin{table}[hp]
\caption{Revision History} \label{TblRevisionHistory}
\begin{tabularx}{\textwidth}{llX}
\toprule
\textbf{Date} & \textbf{Developer(s)} & \textbf{Change}\\
\midrule
Oct 19, 2022 & Everyone & Initial version\\

... & ... & ...\\
\bottomrule
\end{tabularx}
\end{table}

\newpage
\textheight 9in

\section{Introduction}

This document is a hazard analysis of Group 20\textquotesingle s ASL Translator. The ASL Translator is a real-time sign language translation device intended to aid 
individuals who are hard of hearing in day to day communication tasks. This device may also be used for the purpose of facilitating the learning of 
sign language in an educational setting.\\

\section{Scope and Purpose}

The purpose of this document is to identify hazards that may occur when using the ASL Translator specifically in the components, their causes and consequences on user 
operation, hazard mitigation, and their respective requirements.\\

\section{System Boundaries}

The device being referred to in the hazard analysis consists of the following components:

1. A Raspberry Pi\\
2. A camera\\
3. A machine learning model for ASL\\
4. A hand tracking script\\

The system boundaries include the raspberry pi, its machine learning model and hand tracking script, and the camera. Despite the choice of raspberry pi model and camera model being influenced by the user and out of control of the developers, it is important to include these components as they are core to the system’s operation. Therefore, a hazard analysis on these components will still be included.


\section{Definition of Hazard}

The definition of a hazard is based on the definition from Nancy Leveson\textquotesingle s work. A hazard is a property or condition in the system along with a condition in the
environment that results in a loss. A hazard is anything that can cause our system to function incorrectly, or not function at all. In the ASL Translator, there
exists only hazards that affect the proper and intended operation of the device.\\


\section{Critical Assumptions}
At this time, there are no critical assumptions to be made.

\section{Failure Modes and Effects Analysis}

The hazard analysis tool being used is the Failure Modes and Effects Analysis (FMEA). This will enable hazard identification and analysis such that additional safety requirements can 
be created and considered in the implementation of the project.\\

\subsection{Hazards Out of Scope}

The out of scope hazards for our project is primarily based on the user\textquotesingle s decision. This is because we do not have control over the following:\\
\begin{itemize}
    \item The camera that is to be used in conjunction with the system
    \item The Raspberry Pi model and microSD card capacity being used
\end{itemize}

Both components listed above are essential to the functionality of the system. However, there is no enforcement on these aspects as the user may prefer something 
less costly or more costly. The user\textquotesingle s decision towards these components may vary and can affect the overall performance of the system. Steps will be taken to minimize 
the impact of the user\textquotesingle s choice in these categories such as ensuring backwards compatibility is possible with our implementation of code on the Raspberry Pi and camera 
calibration methods.\\

\subsection{Failure Modes and Effects Analysis Table}

Below is the FMEA table for the project.

\tabulinesep=1.2mm
\hspace*{-1cm}\begin{longtabu} to \textwidth {|p{0.1\linewidth}|p{0.125\linewidth}|p{0.125\linewidth}|p{0.125\linewidth}|p{0.255\linewidth}|p{0.05\linewidth}|p{0.05\linewidth}|}
\caption{FMEA Table} \label{TblFMEA} \\
\firsthline
\multicolumn{1}{|c|}{Component} & \multicolumn{1}{c|}{Failure Modes}          & \multicolumn{1}{c|}{Effects of Failure}          & \multicolumn{1}{c|}{Causes of Failure} & \multicolumn{1}{c|}{Recommended Action} & \multicolumn{1}{c|}{Req.} & \multicolumn{1}{c|}{Ref.} \\
\hline \endfirsthead
\hline
\multicolumn{1}{|c|}{Component} & \multicolumn{1}{c|}{Failure Modes}          & \multicolumn{1}{c|}{Effects of Failure}          & \multicolumn{1}{c|}{Causes of Failure} & \multicolumn{1}{c|}{Recommended Action} & \multicolumn{1}{c|}{Req.} & \multicolumn{1}{c|}{Ref.} \\
\hline \endhead
\hline
\multicolumn{7}{r}{Continued on next page} \\
\hline \endfoot
\hline \endlastfoot

\multirow[t]{4}{=}{Raspberry Pi}& Fail to output translated results           & The output is not properly displayed             & a. Code execution failure due to improper flashing to board & a. Reboot the board and ensure that the correct software and dependencies are loaded & a. SR2 & H1-1 \\ \cline{2-7}
                                & \multirow[t]{3}{=}{Hardware failure (board)}& \multirow[t]{3}{=}{Raspberry Pi cannot function} & a. The board is not powered due to faulty power supply & a. Ensure that the board is properly plugged in. Use another power cable to verify that the board itself is not faulty. The raspberry pi is equipped with a polyfuse to prevent over-current. If the board does not power on after 24 hours, the fuse should be replaced. & a. HR1 & \multirow[t]{3}{=}{H1-2}\\ \cline{4-6}
                                &                                             &                                                  & b. The software is corrupted & b. Remove the microSD card and ensure that the card is not corrupted using a computer. If it is corrupted, attempt to reformat the microSD card and write the software onto it again. & b. HR1 & \\ \cline{4-6}
                                &                                             &                                                  & c. The board is faulty or defective & c. Attempt to run the software on an identical model using the same cables, microSD card, and peripherals. If the software works, then the existing board is faulty and should be replaced. Otherwise, test using the newer components until the problem is isolated. & c. HR1 & \\ \cline{4-6}
\multirow[t]{1}{=}{Raspberry Pi}& \multirow[t]{1}{=}{Hardware failure (board)}& \multirow[t]{1}{=}{Raspberry Pi cannot function} & d. The microSD card cannot be read & d. Same as H1-2b & d. HR1 & \multirow[t]{1}{=}{H1-2}\\ \hline
\multirow[t]{4}{=}{Camera}      & \multirow[t]{2}{=}{Failing to detect motion}& \multirow[t]{4}{=}{Device is unable to detect or translate sign language} & a. Poor lighting conditions & a. Ensure the lighting condition in the working environment is sufficient. This will enable the camera to get sufficient light exposure to recognize hand motions. & a. ER1 & \multirow[t]{1}{=}{H2-1}\\ \cline{4-6}
                                &                                             &                                                  & b. Unclear hand gestures by the user & b. The user should make a slightly bigger hand gesture to ensure that the camera is picking the motion. If the problem persists, refer to H2-2a &            & \\ \cline{2-2} \cline{4-7}
                                & \multirow[t]{2}{=}{Hardware failure}        &                                                  & a. Issues with the lenses & a. Ensure that the camera lenses are not clouded by debris and recalibrate the camera. If the problem persists, replace the lenses. & a. HR2 & \multirow[t]{2}{=}{H2-2}\\\cline{4-6}
                                &                                             &                                                  & b. Electrical power issues & b. Make sure that the camera is properly plugged into the system. Test the camera on another device such as a PC, and if the camera is functional. Test the power supply of the board. Refer to H1-2a for power supply testing. & b. HR2 & \\ \hline
\multirow[t]{2}{=}{The machine learning model for ASL} & \multirow[t]{2}{=}{ML model is not accurate} & \multirow[t]{2}{=}{The device will produce incorrect or unexpected outputs} & a. Incorrect data set is being used & a. Double check to see if the input data is correct when training the model & a. MLR1 & \multirow[t]{3}{=}{H3-1}\\ \cline{4-6}
                                &                                             &                                                  & b. ML model is misusing data and turning it into meaningless input & b. Pass irrelevant input through to see if the error is consistent with input that is correct. This indicates that there is an error with how the model is programmed & b. MLR1 & \\ \cline{4-6}
\multirow[t]{2}{=}{The machine learning model for ASL} & \multirow[t]{2}{=}{ML model is not accurate} & \multirow[t]{2}{=}{The device will produce incorrect or unexpected outputs} & c. The training dataset is not randomized enough resulting in the model anticipating specific orders of input & c. Shuffle the data during training to introduce variance into the model to boost output accuracy & c. MLR1 & \multirow[t]{2}{=}{H3-1}\\ \cline{4-6}
                                &                                             &                                                  & d. Model is underfit or overfit resulting in poor generalization outside of the training dataset & d. Same as H3-1c & d. MLR2 & \\ \hline
\multirow[t]{2}{=}{Hand tracking script} & \multirow[t]{2}{=}{Script is not running} & \multirow[t]{2}{=}{The device will not be able to translate sign language} & a. Code execution error & a. Ensure script is error-free and has error handling for foreseeable caes of error & a. SR1 & \multirow[t]{2}{=}{H4-1}\\ \cline{4-6}
                                &                                             &                                                  & b. Code was not flashed onto the board correctly & b. Same as H1-1a & b. SR2 & \\ \hline
\end{longtabu}\hspace*{-1cm}

\section{Requirements}
\subsection{Hardware Requirement}
HR1: The Raspberry Pi board should be checked and tested before using it in the project.\\
HR2: The camera should be checked before each use to ensure the lenses are clean and working properly.
\subsection{Environment Requirement}
ER1: The camera shall be mounted in a manner that prevents obstructions from hindering its operation.
\subsection{Machine Learning Requirement}
MLR1: The dataset chosen should be checked and verified for sufficient information before training. Preprocessing should be done on the dataset for standardization and normalization.\\
MLR2: ML model should be built accordingly for the purposes of classification and tested using a confusion matrix or relevant metrics such as loss and accuracy. Further optimization should be considered such as early stopping or dropout.
\subsection{Script Requirement}
SR1: Various software testing (Black box testing, White box testing, etc) should be performed on the script before it is flashed onto the board.\\
SR2: Complete functionality testing should be performed after the code is flashed onto the board.

\section{Roadmap}
The hazard analysis has resulted in new safety requirements that were not considered in the initial revision of the 
System Requirements Specification (SRS). Given the current time and budget constraints, while some hazards will be accounted 
for and removed, some hazards might still remain in the final revision of the project. These hazards that may not be addressed may include
model inaccuracy and Raspberry Pi hardware failure/failure to display results. This is primarily due to the intensive training needed
for an accurate machine learning model and due to the hardware limitations of a Raspberry Pi as the project develops. In these cases, notices 
will be provided to the user to ensure safety and proper regulation. Throughout the next steps of the project, the hazard analysis will continuously 
be consulted to determine risks that have already been considered in the implementation and risks that will be removed in future updates.

\end{document}
