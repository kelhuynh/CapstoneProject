\documentclass{article}

\usepackage{booktabs}
\usepackage{tabularx}
\usepackage{multirow}
\usepackage{float}
\usepackage{longtable}
\usepackage[none]{hyphenat}
\usepackage[document]{ragged2e}
\usepackage{fullpage}
\usepackage{tabu}

\input{../Comments}
%% Common Parts

\newcommand{\progname}{Mechatronics} % PUT YOUR PROGRAM NAME HERE
\newcommand{\authname}{Team \#20, OpenASL
\\ Robert Zhu zhul49
\\ Zifan Meng mengz17
\\ Jiahui Chen chenj194
\\ Kelvin Huynh huynhk12
\\ Runze Zhu zhur25
\\ Mirza Nafi Hasan hasanm21} % AUTHOR NAMES                  

\usepackage{hyperref}
    \hypersetup{colorlinks=true, linkcolor=blue, citecolor=blue, filecolor=blue,
                urlcolor=blue, unicode=false}
    \urlstyle{same}
                                


\title{Hazard Analysis\\\progname}
\author{\authname}

\date{October 19, 2022}

\begin{document}

\maketitle

~\newpage

\tableofcontents

~\newpage

\section*{Revision History}
\begin{table}[hp]
\caption{Revision History} \label{TblRevisionHistory}
\begin{tabularx}{\textwidth}{llX}
\toprule
\textbf{Date} & \textbf{Developer(s)} & \textbf{Change}\\
\midrule
Date1 & Name(s) & Description of changes\\
Date2 & Name(s) & Description of changes\\
... & ... & ...\\
\bottomrule
\end{tabularx}
\end{table}

\newpage
\textheight 9in

\section{Introduction}

This document is a hazard analysis of Group 20\textquotesingle s ASL Translator. The ASL Translator is a real-time sign language translation device intended to aid 
individuals who are hard of hearing in day to day communication tasks. This device may also be used for the purpose of facilitating the learning of 
sign language in an educational setting.\\

\section{Scope and Purpose}

The purpose of this document is to identify hazards that may occur when using the ASL Translator specifically in the components, their causes and consequences on user 
operation, hazard mitigation, and their respective safety requirements.\\

\section{System Boundaries}

\section{Definition of Hazard}

The definition of a hazard is based on the definition from Nancy Leveson\textquotesingle s work. A hazard is a property or condition in the system along with a condition in the
environment that results in a loss. A hazard is anything that can cause our system to function incorrectly, or not function at all. In the ASL Translator, there
exists only hazards that affect safety during operation.\\


\section{Critical Assumptions}

\section{Failure Modes and Effects Analysis}

The hazard analysis tool being used is the Failure Modes and Effects Analysis (FMEA). This will enable hazard identification and analysis such that additional safety requirements can 
be created and considered in the implementation of the project.\\

\subsection{Hazards Out of Scope}

The out of scope hazards for our project is primarily based on the user\textquotesingle s decision. This is because we do not have control over the following:\\
\begin{itemize}
    \item The camera that is to be used in conjunction with the system
    \item The Raspberry Pi model and microSD card capacity being used
\end{itemize}

Both components listed above are essential to the functionality of the system. However, there is no enforcement on these aspects as the user may prefer something 
less costly or more costly. The user\textquotesingle s decision towards these components may vary and can affect the overall performance of the system. Steps will be taken to minimize 
the impact of the user\textquotesingle s choice in these categories such as ensuring backwards compatibility is possible with our implementation of code on the Raspberry Pi and camera 
calibration methods.\\

\subsection{Failure Modes and Effects Analysis Table}

Below is the FMEA table for the project.\\

\renewcommand{\arraystretch}{2}
\hspace*{-1cm}\begin{longtabu} to \textwidth {|p{0.1\linewidth}|p{0.15\linewidth}|p{0.15\linewidth}|p{0.15\linewidth}|p{0.15\linewidth}|p{0.05\linewidth}|p{0.08\linewidth}|}\firsthline
\toprule
\multicolumn{1}{|c|}{Component} & \multicolumn{1}{c|}{Failure Modes}          & \multicolumn{1}{c|}{Effects of Failure}          & \multicolumn{1}{c|}{Causes of Failure} & \multicolumn{1}{c|}{Recommended Action} & \multicolumn{1}{c|}{SR} & \multicolumn{1}{c|}{Ref.} \\
\hline \endfirsthead
\toprule
\multicolumn{1}{|c|}{Component} & \multicolumn{1}{c|}{Failure Modes}          & \multicolumn{1}{c|}{Effects of Failure}          & \multicolumn{1}{c|}{Causes of Failure} & \multicolumn{1}{c|}{Recommended Action} & \multicolumn{1}{c|}{SR} & \multicolumn{1}{c|}{Ref.} \\
\hline \endhead
\multirow[t]{3}{=}{Raspberry Pi}& Fail to output translated results           &                                                  &                                       &                                        &                        &                          \\ \cline{2-7}
                                & \multirow[t]{2}{=}{Hardware failure (board)}& \multirow[t]{2}{=}{Raspberry Pi cannot function} & a. The board is not powered due to faulty power supply & a. Ensure that the board is properly plugged in. Use another power cable to verify that the board itself is not faulty. The raspberry pi is equipped with a polyfuse to prevent over-current. If the board does not power on after 24 hours, the fuse should be replaced. & a. HR1 & \multirow[t]{2}{=}{H1-2}\\ \cline{4-6}
                                &                                             &                                                  & b. The software is corrupted & b. Remove the microSD card and ensure that the card is not corrupted using a computer. If it is corrupted, attempt to reformat the microSD card and write the software onto it again. & b. HR1 & \\ \hline
\multirow[t]{2}{=}{Raspberry Pi}& \multirow[t]{2}{=}{Hardware failure (board)}& \multirow[t]{2}{=}{Raspberry Pi cannot function} & c. The board is faulty or defective & c. Attempt to run the software on an identical model using the same cables, microSD card, and peripherals. If the software works, then the existing board is faulty and should be replaced. Otherwise, test using the newer components until the problem is isolated. & c. HR1 & \multirow[t]{2}{=}{H1-2}\\ \cline{4-6}
                                &                                             &                                                  & d. The microSD card cannot be read & d. Same as H1-2b & d. HR1                       & \\ \hline                       
\end{longtabu}\hspace*{-1cm}

\section{Safety Requirements}
\subsection{Requirement Category 1}
\subsection{Requirement Category 2}

\section{Roadmap}

\end{document}
