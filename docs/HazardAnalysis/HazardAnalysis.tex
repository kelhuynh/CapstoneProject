\documentclass{article}

\usepackage{booktabs}
\usepackage{tabularx}

%% Comments

\usepackage{color}

\newif\ifcomments\commentstrue %displays comments
%\newif\ifcomments\commentsfalse %so that comments do not display

\ifcomments
\newcommand{\authornote}[3]{\textcolor{#1}{[#3 ---#2]}}
\newcommand{\todo}[1]{\textcolor{red}{[TODO: #1]}}
\else
\newcommand{\authornote}[3]{}
\newcommand{\todo}[1]{}
\fi

\newcommand{\wss}[1]{\authornote{blue}{SS}{#1}} 
\newcommand{\plt}[1]{\authornote{magenta}{TPLT}{#1}} %For explanation of the template
\newcommand{\an}[1]{\authornote{cyan}{Author}{#1}}

%% Common Parts

\newcommand{\progname}{Mechatronics} % PUT YOUR PROGRAM NAME HERE
\newcommand{\authname}{Team \#20, Team Name
\\ Robert Zhu zhul49
\\ Zifan Meng mengz17
\\ Jiahui Chen chenj194
\\ Kelvin Huynh huynhk12
\\ Runze Zhu zhur25
\\ Mirza Nafi Hasan hasanm21} % AUTHOR NAMES                  

\usepackage{hyperref}
    \hypersetup{colorlinks=true, linkcolor=blue, citecolor=blue, filecolor=blue,
                urlcolor=blue, unicode=false}
    \urlstyle{same}
                                


\title{Hazard Analysis\\\progname}
\author{\authname}

\date{October 19, 2022}

\begin{document}

\maketitle

~\newpage

\tableofcontents

~\newpage

\section*{Revision History}
\begin{table}[hp]
\caption{Revision History} \label{TblRevisionHistory}
\begin{tabularx}{\textwidth}{llX}
\toprule
\textbf{Date} & \textbf{Developer(s)} & \textbf{Change}\\
\midrule
Date1 & Name(s) & Description of changes\\
Date2 & Name(s) & Description of changes\\
... & ... & ...\\
\bottomrule
\end{tabularx}
\end{table}

\newpage

\section{Introduction}

This document is a hazard analysis of Group 20\textquotesingle s ASL Translator. The ASL Translator is a real-time sign language translation device intended to aid 
individuals who are hard of hearing in day to day communication tasks. This device may also be used for the purpose of facilitating the learning of 
sign language in an educational setting.\\

\section{Scope and Purpose}

The purpose of this document is to identify hazards that may occur when using the ASL Translator specifically in the components, their causes and consequences on user 
operation, hazard mitigation, and their respective safety requirements.\\

\section{System Boundaries}

\section{Definition of Hazard}

The definition of a hazard is based on the definition from Nancy Leveson’s work. A hazard is a property or condition in the system along with a condition in the
environment that results in a loss. A hazard is anything that can cause our system to function incorrectly, or not function at all. In the ASL Translator, there
exists only hazards that affect safety during operation.\\


\section{Critical Assumptions}

\section{Failure Modes and Effects Analysis}

\subsection{Hazards Out of Scope}

\subsection{Failure Modes and Effects Analysis Table}

\section{Safety Requirements}
\subsection{Hareware Requirement}
HR1: The Raspberry Pi board should be checked and tested before using it in the project.
HR2: The camera should be checked before each use to ensure the lenses are clean and working properly.

\subsection{Requirement Category 2}

\section{Roadmap}

\end{document}
